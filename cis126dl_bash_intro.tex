\documentclass[12pt,handout,aspectratio=169]{beamer}
\usepackage{alltt}
\usepackage{graphicx}
\usepackage{hyperref}
\usepackage[latin1]{inputenc}
\usepackage{listings}
\usepackage{relsize}
\usepackage{tikz}
\usepackage{todonotes}
\mode<presentation>
{
  \usetheme{Frankfurt}
  \usecolortheme{seahorse}
  \useoutertheme{infolines}
  \useinnertheme{rounded}
  \transdissolve
  \setbeamercovered{transparent}
  \hypersetup{colorlinks=true,linkcolor=blue}
  \graphicspath{ {./images/} }
%  \transdissolve<2>
  \beamerdefaultoverlayspecification{<+->}
  \beamertemplatenavigationsymbolsempty
}
%\mode<presentation>
%\mode<presentation>{ \usetheme{boxes} \usetheme{Berlin} }
%\useoutertheme{shadow}
%\setbeamertemplate{blocks}[rounded][shadow=true]
%\usetheme{Frankfurt}
%\usecolortheme{seahorse}
%\usefonttheme[onlylarge]{structuresmallcapsserif}
%\usefonttheme[onlysmall]{structurebold}
%\setbeamercolor{title}{fg=red!80!black,bg=red!20!white}
%Information to be included in the title page:
\title{Introduction to Shell Scripting}
\subtitle{or how to be a shell Ninja}
\author{Dennis Kibbe \\\href{mailto:dennis.kibbe@mesacc.edu}{dennis.kibbe@mesacc.edu}}
\institute{Mesa Community College \\Network Academy}
\date{Fall 2016}
\subject{Computer Information Systems}
\begin{document}
\maketitle
\begin{frame}
\frametitle{Table of Contents}
\tableofcontents[currentsection]
\end{frame}
\addtocounter{framenumber}{-1}
\section{Parts of a Script}
\begin{frame}
  \frametitle{Why Script?}
  \framesubtitle{A bit more information about this}
  % Content goes here
  Scripting lets system administrators:
  \begin{itemize}
  \item Combine commands to accomplish a task.
    \pause
  \item Automate repetitive processes.
    \pause
  \item Branch after testing for a condition.
    \pause
  \item Improve efficiency.
    \pause
  \end{itemize}
\end{frame}
\begin{frame}[fragile]
  \frametitle{System Files}
  \framesubtitle{Bash shell scripts start many system processes.}
  \begin{block}{Starts the rsyslog Daemon}
    \begin{verbatim}
NAME=rsyslog

# Exit if the package is not installed
[ -x "$DAEMON" ] || exit 0

# Read configuration variable file if it is present
[ -r /etc/default/$NAME ] && . /etc/default/$NAME
\end{verbatim}
\end{block}
\end{frame}
\begin{frame}[fragile]
  \frametitle{Which Shell?}
  \framesubtitle{Finding your default shell}
  \begin{block}{The SHELL Variable}
    \begin{verbatim}
      $ env | grep SHELL
      SHELL=/bin/bash
    \end{verbatim}
  \end{block}
\end{frame}

\section{Which Shell Are You Using?}

\begin{frame}[fragile]
  \frametitle{Which Shell?}
  \framesubtitle{Finding your default shell}
  \begin{block}{/etc/passwd}
    \begin{verbatim}
      tux:x:1000:1000:Tux Penguin,,,:/home/tux:/bin/bash
    \end{verbatim}
  \end{block}
\end{frame}
\begin{frame}[fragile]
  \frametitle{Programming At The Command Line}
  You can enter programming commands directly at the command line.
   \begin{block}{At the Command Line}
    \begin{verbatim}
      $ for i in {1..5}
      > do touch file\$i
      > done
    \end{verbatim}
   \end{block}
\end{frame}

\section{Using the Correct Editor}

\begin{frame}
  \frametitle{Editing Scripts}
  Shell scripts must be edited in a text editor that saves the script as plain, unformatted ASCII text. Common editors are:
  \begin{itemize}
  \item Vi/Vim
  \item Emacs
  \item Nano
  \item Gedit
  \item notepad++
  \end{itemize}
\end{frame}
\begin{frame}
  \frametitle{Editing Scripts}
  Editors not to use are any word processor such as:
  \begin{itemize}
  \item Libre Office
  \item OpenOffice.org
  \item M\$ Word
  \end{itemize}
\end{frame}
\begin{frame}[fragile]
  \frametitle{Line Endings}
  Windows uses both a carriage return and a new line character.  Linux and Unix use new line only.  Editing a script in Windows notepad will break the script.
  \begin{block}{Line endings}
\begin{verbatim}
#!/bin/bash^M
    # This script was edited in Windows Notepad^M
    ^M
    echo "Hello World"^M
    ^M
\end{verbatim}
  \end{block}
\end{frame}
\section{Parts of a Script}

\begin{frame}
  \frametitle{Anatomy Of A Shell Script}
  Common parts of a shell script are:
  \begin{itemize}
  \item Shebang
  \item Comments
  \item Variables
  \item Functions
  \item Exacutable code
  \end{itemize}
\end{frame}
\begin{frame}[fragile]
  \frametitle{Shebang}
  The Shebang line assures that the script is run as a Bash script and must be the first line in the script.
\begin{block}{}
\begin{verbatim}
#!/bin/bash
\end{verbatim}
\end{block}
\end{frame}
\begin{frame}[fragile]
  \frametitle{Shebang (Cont.)}
  The Shebang line assures that the script is run as a Bash script and must be the first line in the script.
\begin{block}{}
\begin{verbatim}
#!/usr/bin/env bash
\end{verbatim}
\end{block}
\end{frame}
\begin{frame}[fragile]
  \frametitle{Comments}
  Any text after the hash tag character \# is a comment. Here is an example from ~/.bashrc.
\begin{block}{Comments}
\begin{verbatim}
# uncomment for a colored prompt, if the terminal has the capability; turned
	# off by default to not distract the user: the focus in a terminal window
	# should be on the output of commands, not on the prompt
	force_color_prompt=yes # This is also a comment.
\end{verbatim}
\end{block}
\end{frame}
\begin{frame}[fragile]
  \frametitle{Variables}
  Defining variables in your script will make it easier to modify.
\begin{block}{}
  \begin{verbatim}
	#!/usr/bin/env bash

	# Set message to print.
	message="Hello world"

	# Print the value of message to STDOUT.

	echo "$message"

	# END
  \end{verbatim}
\end{block}
\end{frame}
\begin{frame}[fragile]
  \frametitle{Variables (Cont.)}
  The dollar sign in from of the variable name tells echo to print the value of the variable and not just the word message.
\begin{block}{}
\begin{verbatim}
$ echo $message
\end{verbatim}
\end{block}
\end{frame}
\begin{frame}[fragile]
  \frametitle{Functions}
  Functions are mini-scripts you can call inside your script. First define the function...
\begin{block}{}
\begin{verbatim}
#!/usr/bin/env bash

# define usage function

usage(){
echo "Usage: $0 greeting"
exit 1
}
\end{verbatim}
\end{block}
\end{frame}
\begin{frame}[fragile]
  \frametitle{Functions (Cont.)}
  then call it when needed.
\begin{block}{}
\begin{verbatim}
	if [[ ! $1 ]]; then
	usage
	else
	echo "$1"
	fi

	# END
\end{verbatim}
\end{block}
\end{frame}
\begin{frame}[fragile]
  \frametitle{Conditionals}
  Depending on the outcome of a test something is done. Example from ~/.bashrc.
\begin{block}{}
\begin{verbatim}
	f# Alias definitions.
	# You may want to put all your additions into a separate file like
	# ~/.bash_aliases, instead of adding them here directly.
	# See /usr/share/doc/bash-doc/examples in the bash-doc package.

		if [ -f ~/.bash_aliases ]; then
		    . ~/.bash_aliases
		fi
\end{verbatim}
\end{block}
\end{frame}
\begin{frame}[fragile]
  \frametitle{Code}
  Finally the code that does the work.
\begin{verbatim}
$ echo "Hello World"
\end{verbatim}

\end{frame}
\section{Feeding Variables to a script}

\begin{frame}[fragile]
  \frametitle{Positional Parameter}
  Positional Parameters let you feed arguments to the script. Use quotes around multiple words.
\begin{block}{}
\begin{verbatim}
$ myscript.sh Hello World
\end{verbatim}
\end{block}
\end{frame}
\begin{frame}[fragile]
  \frametitle{Positional Parameter (Cont.)}
  Each positional parameters is identified by a number 1 to 9.
\begin{block}{}
\begin{verbatim}
#!/usr/bin/env bash

# Prints the value of the first two positional parameters.

echo "$1" "$2"

# END
\end{verbatim}
\end{block}
\end{frame}
\begin{frame}[fragile]
  \frametitle{For Loop}
  Do iterations over a list.
\begin{verbatim}
$ for i in apple banana cherry "red grapes" ; do echo "$i"; done
\end{verbatim}
\end{frame}
\begin{frame}[fragile]
  \frametitle{If/Then/Else}
\begin{block}{}
\begin{verbatim}
if [[ ! $1 ]]
then echo "I have nothing to print."
else
echo "$1"
fi
\end{verbatim}
\end{block}
\end{frame}
\begin{frame}[fragile]
  \frametitle{Conditionals}
\begin{verbatim}
if [[ ! $1 ]]; then
echo "I have nothing to print."
else
echo "$1"
fi
\end{verbatim}
\end{frame}
\begin{frame}[fragile]
  \frametitle{Variables}
  Variable are easy to set.
\begin{verbatim}
message="Hello World"
echo $message
\end{verbatim}

\end{frame}
\begin{frame}[fragile]
  \frametitle{Variables (Cont.)}
You can used pre-definded environmental variables, too.
\begin{verbatim}
  echo "Today is $(date +%A) and you are logged into $HOSTNAME as $USER."
\end{verbatim}

\end{frame}
\begin{frame}[fragile]
  \frametitle{While Loops}
  \begin{verbatim}
    #!/bin/bash

# Set inital value for variable
counter=0

while [ $counter -lt 10 ]; do
echo "The counter is $counter"
sleep 1
let counter=$counter+1
done

# END
  \end{verbatim}
\end{frame}
\section{Exit Codes}

\begin{frame}[fragile]
  \frametitle{Exit Code}
  Every program returns an exit code when it finishes. An exit code of zero means that the command succeeded.
\begin{verbatim}
$ ls /tmp
$ echo $?
0
\end{verbatim}
\end{frame}
\begin{frame}[fragile]
  \frametitle{Exit Code (Cont.)}
  Every program returns an exit code when it finishes. An exit code other than zero means that the command failed.
\begin{verbatim}
$ ls /temp
ls: cannot access /temp: No such file or directory
$ echo $?
1
\end{verbatim}

\end{frame}
\section{Finding help}

\begin{frame}[fragile]
  \frametitle{Getting Help}
\begin{verbatim}
help test
test: test [expr]
Evaluate conditional expression.

Exits with a status of 0 (true) or 1 (false) depending on the evaluation of EXPR.  Expressions may be unary or binary.  Unary expressions are often used to examine the status of a file.  There are string operators and numeric comparison operators as well.

The behavior of test depends on the number of arguments.  Read the bash manual page for the complete specification.
\end{verbatim}

\end{frame}
\begin{frame}
  \frametitle{Resources}
  \begin{itemize}
  \item \href{http://www.shellcheck.net/}{Shell Check}
  \item \href{http://mywiki.wooledge.org/BashGuide}{Greg's Bash Guide}
  \item \href{http://explainshell.com/}{Shell Explained}
  \item \href{https://google.github.io/styleguide/shell.xml}{Google Style Guide}
    \item \href{https://kb.iu.edu/d/acux}{How do I convert between Unix and Windows text files?}

    
  \end{itemize}
\end{frame}

\end{document}