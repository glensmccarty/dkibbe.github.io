\documentclass[12pt,handout,aspectratio=169]{beamer}
\usepackage{fancybox}
\usepackage{graphicx}
\usepackage{hyperref}
\usepackage[latin1]{inputenc}
\usepackage{tikz}
\usepackage{todonotes}
\mode<presentation>
{
  \usetheme{AnnArbor}
  \usecolortheme{seahorse}
  \useoutertheme{infolines}
  \useinnertheme{rounded}
  \transdissolve
  \setbeamercovered{transparent}
  \hypersetup{colorlinks=true,linkcolor=blue}
  \graphicspath{ {./images/} }
%  \transdissolve<2>
  \beamerdefaultoverlayspecification{<+->}
  \beamertemplatenavigationsymbolsempty
}
%\pgfpagesuselayout{2 on 1}[a4paper,border shrink=5mm]
%\usefonttheme[onlylarge]{structuresmallcapsserif}
%\usefonttheme[onlysmall]{structurebold}
%\setbeamercolor{title}{fg=red!80!black,bg=red!20!white}
%Information to be included in the title page:
\title{Su \& Sudo}
\subtitle{Working as root}
\author{Dennis Kibbe\\\href{mailto:dennis.kibbe@mesacc.edu}{dennis.kibbe@mesacc.edu}}
\institute{Mesa Community College\\Network Academy}
\date{Fall 2016}
\subject{Computer Information Systems}
\begin{document}
\maketitle
\begin{frame}
\frametitle{Table of Contents}
\tableofcontents[currentsection]
\end{frame}
\begin{frame}
\frametitle{Su \& Sudo}
The su and sudo commands are used to run commands as root on a Linux system. Some distributions have sudo installed and active by default others don't and use su.  We'll look at both and examine the advantages of sudo over su.
\end{frame}
\begin{frame}[fragile]
\frametitle{Second Slide}
The su command will switch to another user if that user's password is known.
\end{frame}
\begin{frame}[fragile]
  \frametitle{SU Example}
\begin{verbatim}
tux@instr:~]$ su - jstudent
Password: (Enter password for jstudent)
jstudent@intr:~]$ pwd
/home/jstudent
\end{verbatim}
\end{frame}
\begin{frame}[fragile]
  \frametitle{Login Shell}
If the user login isn't given root is assumed. The -l option assures that the complete path for the user is used.
\begin{verbatim}
tux@instr:~]$ su -l
Password: (Enter password for root)
root@intr:~]# pwd
/root
\end{verbatim}
\end{frame}
\begin{frame}[fragile]
  \frametitle{Login Shell}
\begin{verbatim}
jstudent@intr:~]$ su
Password:
root@intr:~]# echo $PATH
	  /usr/local/bin:/usr/bin:/bin:/usr/games:/home/jstudent/bin
root@intr:/home/jstudent]# pwd
/home/jstudent
\end{verbatim}
\end{frame}
\begin{frame}[fragile]
  \frametitle{login Shell}
\begin{verbatim}
jstudent@intr:~]$ su -
Password:
root@intr:~]# echo $PATH
	  /usr/local/bin:/usr/bin:/bin:/sbin:/usr/sbin:/root/bin
root@intr:~]# pwd
/root
\end{verbatim}
\end{frame}
\begin{frame}
  \frametitle{Su - Disadvantages}
  \begin{itemize}
  \item The root password must be shared.
  \item Full root privileges are given.
  \item There is no audit trail.
  \item Managing many root users is hard.
  \end{itemize}
\end{frame}
\begin{frame}
  \frametitle{User Privilege Lines}
  \begin{block}{}
\begin{semiverbatim}
\textbf{root} ALL=(ALL:ALL) ALL
\end{semiverbatim}
  \end{block}
\end{frame}
\begin{frame}
  \frametitle{User Privilege Lines}
  \begin{block}{}
\begin{semiverbatim}
root \textbf{ALL}=(ALL:ALL) ALL
\end{semiverbatim}
  \end{block}
\end{frame}
\begin{frame}
  \frametitle{Sudo}
    %\includegraphics[height=3cm]{sudo_penguin.jpg}
\end{frame}
\begin{frame}
  \frametitle{Sudo Advantages}
  \begin{itemize}
  \item The root account can be locked.
  \item Sudoers use their own password.
  \item Fine-grained permissions are possible.
  \item All activity is logged.
  \item Managing sudoers is easy.
  \end{itemize}
\end{frame}
\begin{frame}
  \frametitle{Sudo configuration}
  /etc/sudoers is the main configuration file.  The must be edited using visudo.  A mistake editing the file may require booting from a rescue disk!
\end{frame}
\begin{frame}
  \frametitle{Aliases and Specifications}
  \begin{itemize}
  \item Aliases
  \item User Specifications
  \end{itemize}
\end{frame}
\begin{frame}
  \frametitle{Aliases}
  \begin{itemize}
  \item User\_Alias
  \item Runas\_Alias
  \item Host\_Alias
  \item Cmnd\_Alias
  \end{itemize}
\end{frame}
\begin{frame}[fragile]
  \frametitle{User Alias}
User\_Alias defines groups of users
\begin{verbatim}
User_Alias	FULLTIMERS = millert, mikef, dowdy
User_Alias	PARTTIMERS = bostley, jwfox, crawl
User_Alias	WEBMASTERS = will, wendy, wim
\end{verbatim}
\end{frame}
\begin{frame}[fragile]
  \frametitle{Runas Alias}
  Runas\_Alias defines how processes are run.
\begin{verbatim}
Runas_Alias	OP = root, operator
Runas_Alias	DB = oracle, sybase
\end{verbatim}
\end{frame}
\begin{frame}[fragile]
  \frametitle{Examples}
\begin{semiverbatim}
root		ALL = (ALL) ALL
%wheel		ALL = (ALL) ALL
\end{semiverbatim}
\end{frame}
\end{document}