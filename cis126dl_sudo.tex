\documentclass[12pt,handout,aspectratio=169]{beamer}
\usepackage{fancybox}
\usepackage{graphicx}
\usepackage{hyperref}
\usepackage[latin1]{inputenc}
\usepackage{tikz}
\usepackage{todonotes}
\mode<presentation>
{
  \usetheme{AnnArbor}
  \usecolortheme{seahorse}
  \useoutertheme{infolines}
  \useinnertheme{rounded}
  \transdissolve
  \setbeamercovered{transparent}
  \hypersetup{colorlinks=true,linkcolor=blue}
  \graphicspath{ {./images/} }
%  \transdissolve<2>
  \beamerdefaultoverlayspecification{<+->}
  \beamertemplatenavigationsymbolsempty
}
%\pgfpagesuselayout{2 on 1}[a4paper,border shrink=5mm]
%\usefonttheme[onlylarge]{structuresmallcapsserif}
%\usefonttheme[onlysmall]{structurebold}
%\setbeamercolor{title}{fg=red!80!black,bg=red!20!white}
%Information to be included in the title page:
\title{Su \& Sudo}
\subtitle{Working as root}
\author{Dennis Kibbe\\\href{mailto:dennis.kibbe@mesacc.edu}{dennis.kibbe@mesacc.edu}}
\institute{Mesa Community College\\Network Academy}
\date{Fall 2016}
\subject{Computer Information Systems}
\begin{document}
\maketitle
\begin{frame}
\frametitle{Table of Contents}
\tableofcontents[currentsection]
\end{frame}
\begin{frame}
\frametitle{First Slide}
\framesubtitle{A bit more information about this}
%Content goes here
Contents of the first slide
\end{frame}
\begin{frame}[fragile]
\frametitle{Second Slide}
%Content goes here
Contents of the second slide
\end{frame}
\begin{frame}
  \frametitle{User Privilege Lines}
  \begin{block}{}
\begin{semiverbatim}
\textbf{root} ALL=(ALL:ALL) ALL
\end{semiverbatim}
  \end{block}
\end{frame}
\begin{frame}
  \frametitle{User Privilege Lines}
  \begin{block}{}
\begin{semiverbatim}
root \textbf{ALL}=(ALL:ALL) ALL
\end{semiverbatim}
  \end{block}
\end{frame}
\begin{frame}[fragile]
  \frametitle{Examples}
\begin{semiverbatim}
root		ALL = (ALL) ALL
%wheel		ALL = (ALL) ALL
\end{semiverbatim}
\end{frame}
\end{document}